\documentclass{EPUProjetDi}

\makeindex

%remplir les lignes suivantes avec les informations vous concernant :
\title[Titre court du projet]{Deep-Agora}

\projet{R\&D Project Report 2022-2023}

\author{Théo Boisseau\\ %Attention : toujours écrire d'abord le prénom puis le nom (ne pas mettre tout le nom en majuscules)
\noindent[\url{theo.boisseau@etu.univ-tours.fr}]
}

\encadrant{Jean-Yves Ramel\\ %
\noindent[\url{jean-yves.ramel@univ-tours.fr}]\\~\\
Polytech Tours\\
Département Informatique %
}

%%%%%%%%%%%%%%%%%%%%%%%%%%%%%%%%%%%%%%%%%%%%%%%%%%%%%%%%%%%%%%%%%%%%%%%%%%%%%
%--------------------------------------------------------------------------------
\begin{document}

\maketitle

\pagenumbering{roman}
\setcounter{page}{0}

{
%on réduit momentanément l'écart entre paragraphes pour ne pas trop aérer la table des matières
\setlength{\parskip}{0em}

\tableofcontents

%\listoffigures
%rq1 : si vous n'avez pratiquement pas de figures, laissez la ligne précédente en commentaire

%\listoftables
%rq1 : si vous n'avez pratiquement pas de tables, laissez la ligne précédente en commentaire
}


\start
%%%%%%%%%%%%%%%%%%%%%%%%%%%%%%%%%%%%%%%%%%%%%%%%%%%%%%%%%%%%%%%%%%%%%%%%%%%%%
%--------------------------------------------------------------------------------
\chapter*{Acknowledgements}
\addcontentsline{toc}{chapter}{\numberline{}Acknowledgements}
\markboth{\hspace{0.5cm}Acknowledgements}{}

.


%%%%%%%%%%%%%%%%%%%%%%%%%%%%%%%%%%%%%%%%%%%%%%%%%%%%%%%%%%%%%%%%%%%%%%%%%%%%%
%--------------------------------------------------------------------------------
\chapter*{Introduction}
%le 2 lignes suivantes permettent d'ajouter l'introduction à la table des matières
%et d'afficher "Introduction en haut des pages"
\addcontentsline{toc}{chapter}{\numberline{}Introduction}
\markboth{\hspace{0.5cm}Introduction}{}

.


%%%%%%%%%%%%%%%%%%%%%%%%%%%%%%%%%%%%%%%%%%%%%%%%%%%%%%%%%%%%%%%%%%%%%%%%%%%%%
%--------------------------------------------------------------------------------
\chapter{Chapter}

.



%%%%%%%%%%%%%%%%%%%%%%%%%%%%%%%%%%%%%%%%%%%%%%%%%%%%%%%%%%%%%%%%%%%%%%%%%%%%%
%--------------------------------------------------------------------------------
\chapter*{Conclusion}
\addcontentsline{toc}{chapter}{\numberline{}Conclusion}
\markboth{Conclusion}{}\label{sec:conclusion}

.



%%%%%%%%%%%%%%%%%%%%%%%%%%%%%%%%%%%%%%%%%%%%%%%%%%%%%%%%%%%%%%%%%%%%%%%%%%%%%
%--------------------------------------------------------------------------------
%exemple de bibliographie
\begin{thebibliography}{99}
\label{sec:biblio}
\bibitem[USP]{finance}	. \url{.}
\end{thebibliography}


%--------------------------------------------------------------------------------
%si on donne des annexes :
\appendix
\addcontentsline{toc}{part}{\numberline{}Annexes}


%--------------------------------------------------------------------------------

\chapter{}

.


%--------------------------------------------------------------------------------

\chapter{Useful links\label{sec:liens_utiles}}
Here is a small list of interesting urls about this project:

\begin{enumerate}
\item \href{www.polytech.univ-tours.fr}{Website of Polytech Tours}
\end{enumerate}


%--------------------------------------------------------------------------------
%index : attention, le fichier dindex .ind doit avoir le même nom que le fichier .tex
%\printindex

%--------------------------------------------------------------------------------
%page du dos de couverture :

\resume{
5 lignes
}

\motcles{5-10 mots clés}

\abstract{
5 lines
}

\keywords{5-10 key words}


\makedernierepage


\end{document}
%%FIN du fichier